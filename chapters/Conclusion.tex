\section{Conclusion}
Throughout my internship at Statcon Electronics India Ltd., I gained extensive
knowledge and practical experience in implementing Space Vector Pulse Width
Modulation (SVPWM). This journey involved understanding and applying various
mathematical transformations, developing control algorithms, and transitioning
from simulation to real hardware implementation.

\noindent
SVPWM offers several advantages over conventional techniques, such as:
\begin{itemize}
    \item \textbf{Higher Efficiency}: SVPWM maximizes the DC bus voltage utilization, leading to improved efficiency in inverter operation. This is crucial for applications where energy efficiency is paramount, such as in renewable energy systems and electric vehicles.
    \item \textbf{Reduced Harmonic Distortion}: By optimizing the switching sequences, SVPWM reduces the harmonic distortion in the output waveforms. This results in cleaner and more stable power delivery, which is essential for sensitive electronic equipment and improving the overall power quality.
    \item \textbf{Improved Power Factor}: SVPWM enables better control over the output voltage and current, contributing to an improved power factor. This is beneficial in reducing energy losses and ensuring that power systems operate closer to their maximum efficiency.
\end{itemize}

\noindent
Moreover, the hands-on experience with simulation tools like MATLAB/Simulink
and practical implementation using STM32F411 microcontrollers has enriched my
understanding of both software and hardware aspects of power electronics. This
dual exposure has equipped me with a holistic view of the challenges and
solutions in modern inverter technology.

\section{Future Scope}
The knowledge and skills developed during this internship have broad
applications in various fields, including:

\begin{itemize}
    \item \textbf{Hybrid Inverters}: Enhancing the efficiency and performance of hybrid inverters. By integrating SVPWM, hybrid inverters can achieve better efficiency and reliability, making them more viable for both residential and commercial energy solutions.
    \item \textbf{Active Front Ends}: Driving DC series motors in trains for power factor correction and voltage regulation. Implementing SVPWM in active front-end converters can lead to significant improvements in power quality and energy savings in traction systems.
    \item \textbf{Battery Charging}: Implementing SVPWM for efficient battery charging systems. With the growing demand for electric vehicles and renewable energy storage, efficient battery charging solutions are critical. SVPWM can contribute to faster and more efficient charging cycles, prolonging battery life and enhancing performance.
    \item \textbf{Electric Vehicle (EV) Drive Systems}: Applying SVPWM in EV drive systems to improve motor efficiency and reduce energy consumption. This can lead to extended range and better performance of electric vehicles, making them more competitive with traditional internal combustion engine vehicles.
\end{itemize}

\noindent
Overall, this internship has been a valuable learning experience, providing me
with the technical expertise and practical skills needed to contribute
effectively to the field of power electronics and embedded systems. The
insights gained have not only broadened my knowledge base but also ignited a
passion for further exploration and innovation in this dynamic and impactful
field.