\section{Simulation}
Before initiating the implementation phase, it was imperative to conduct a
comprehensive simulation of the front end using Simulink/MATLAB. The simulation
setup involved importing a three-phase source from MATLAB and configuring the
necessary parameters for voltage and current measurements.

\subsection{MATLAB Function Blocks in Simulation}
To emulate the behavior of a microcontroller in the simulation environment,
MATLAB function blocks were extensively utilized. These function blocks
provided a familiar coding environment, resembling the programming paradigms
employed in microcontroller firmware development. By encapsulating custom
algorithms and control logic within MATLAB function blocks, it was possible to
simulate complex control strategies and signal processing techniques with ease.
This approach not only facilitated rapid prototyping and testing but also
provided invaluable insights into the real-time behavior of the system. The use
of MATLAB function blocks bridged the gap between simulation and
implementation, enabling seamless transition from design validation to hardware
deployment.

\subsection{Voltage Transformation}
The focus was on transforming the measured line voltages from the three-phase
source into the alpha-beta frame using the Clarke transform. This
transformation facilitated easier analysis and control of the three-phase
system.

\subsection{Phase-Locked Loop (PLL)}
Incorporating a Phase-Locked Loop (PLL) block was essential for converting the
voltages from the alpha-beta frame to the dq rotating reference frame.
Additionally, the PLL provided the angle theta of the voltage vector, crucial
for subsequent control strategies.

\subsection{Current Measurement and Transformation}
Here, the line current flowing into the front end was measured, and using
Clarke and Park transform blocks, it was converted to the dq reference frame.
The angle theta obtained from the PLL was utilized for this transformation.

\subsection{Current Regulation}
Regulating the current flowing into the front end was pivotal for ensuring
stable operation and adherence to system constraints. Two PI controllers were
employed to compare the measured current to a predetermined set current value
and generate control signals for current regulation.

\subsection{Voltage Calculation and Transformation}
This subsection involved adding the control signals (delta Vq and delta Vd)
obtained from the PI controllers to the Vq and Vd components obtained from the
PLL. These voltages were then converted back to the Clarke frame using the
inverse Park transform.

\subsection{Gate Timing Calculation}
Utilizing the voltages calculated in the previous step (V alpha * and V beta
*), gate timings for the three-phase active front end were determined. These
gate timings dictated the switching patterns of the IGBTs, essential for
controlling the flow of current through the front end.

\section{implementation}